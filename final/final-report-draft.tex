% Options for packages loaded elsewhere
\PassOptionsToPackage{unicode}{hyperref}
\PassOptionsToPackage{hyphens}{url}
%
\documentclass[
  twocolumn]{article}
\usepackage{lmodern}
\usepackage{amssymb,amsmath}
\usepackage{ifxetex,ifluatex}
\ifnum 0\ifxetex 1\fi\ifluatex 1\fi=0 % if pdftex
  \usepackage[T1]{fontenc}
  \usepackage[utf8]{inputenc}
  \usepackage{textcomp} % provide euro and other symbols
\else % if luatex or xetex
  \usepackage{unicode-math}
  \defaultfontfeatures{Scale=MatchLowercase}
  \defaultfontfeatures[\rmfamily]{Ligatures=TeX,Scale=1}
\fi
% Use upquote if available, for straight quotes in verbatim environments
\IfFileExists{upquote.sty}{\usepackage{upquote}}{}
\IfFileExists{microtype.sty}{% use microtype if available
  \usepackage[]{microtype}
  \UseMicrotypeSet[protrusion]{basicmath} % disable protrusion for tt fonts
}{}
\makeatletter
\@ifundefined{KOMAClassName}{% if non-KOMA class
  \IfFileExists{parskip.sty}{%
    \usepackage{parskip}
  }{% else
    \setlength{\parindent}{0pt}
    \setlength{\parskip}{6pt plus 2pt minus 1pt}}
}{% if KOMA class
  \KOMAoptions{parskip=half}}
\makeatother
\usepackage{xcolor}
\IfFileExists{xurl.sty}{\usepackage{xurl}}{} % add URL line breaks if available
\IfFileExists{bookmark.sty}{\usepackage{bookmark}}{\usepackage{hyperref}}
\hypersetup{
  pdftitle={Bike-Sharing Data Analysis: Prediction of Daily Bike Rental Counts Based on Multiple Linear Regression},
  pdfauthor={Ali Taqi, Hsin-Chang Lin, Huiru Yang, Ryan Mahoney, Yulin Li},
  hidelinks,
  pdfcreator={LaTeX via pandoc}}
\urlstyle{same} % disable monospaced font for URLs
\usepackage[margin=1in]{geometry}
\usepackage{graphicx,grffile}
\makeatletter
\def\maxwidth{\ifdim\Gin@nat@width>\linewidth\linewidth\else\Gin@nat@width\fi}
\def\maxheight{\ifdim\Gin@nat@height>\textheight\textheight\else\Gin@nat@height\fi}
\makeatother
% Scale images if necessary, so that they will not overflow the page
% margins by default, and it is still possible to overwrite the defaults
% using explicit options in \includegraphics[width, height, ...]{}
\setkeys{Gin}{width=\maxwidth,height=\maxheight,keepaspectratio}
% Set default figure placement to htbp
\makeatletter
\def\fps@figure{htbp}
\makeatother
\setlength{\emergencystretch}{3em} % prevent overfull lines
\providecommand{\tightlist}{%
  \setlength{\itemsep}{0pt}\setlength{\parskip}{0pt}}
\setcounter{secnumdepth}{5}

\title{Bike-Sharing Data Analysis: Prediction of Daily Bike Rental Counts Based
on Multiple Linear Regression}
\usepackage{etoolbox}
\makeatletter
\providecommand{\subtitle}[1]{% add subtitle to \maketitle
  \apptocmd{\@title}{\par {\large #1 \par}}{}{}
}
\makeatother
\subtitle{Final Project Report · MA 575 Fall 2021 · C3 · Team \#2}
\author{Ali Taqi, Hsin-Chang Lin, Huiru Yang, Ryan Mahoney, Yulin Li}
\date{12/10/2021}

\begin{document}
\maketitle

In this project, the following question is to be answered: If we have
the past history of bike rental counts as well as records of
environmental and seasonal conditions, how and how well could we predict
the bike rental counts in the future? In this project, such questions
are approached by predictive modeling of daily bike rental counts from a
2011-2012 Bike Sharing dataset {[}1{]}. The daily bike rental counts are
predicted with models based on Multiple Linear Regression (MLS) using
the environmental and seasonal variables as predictors. The initial goal
of this project is to train the model using only the 2011 data, and then
validate the prediction power of the model on the 2012 data. Given the
limited time span of available training data, issues are found in the
validation process using the 2012 data; the impact of user base on the
future predictions is brought to our attention. The initial models are
then revisited and corrected to account for the effect of user base. The
refined models are expected to have better prediction powers than the
initial MLS models, but a full validation would require further
availability of bike rental data.

\hypertarget{introduction}{%
\section{Introduction}\label{introduction}}

Bike sharing has become a world-wide phenomenon. Optimization of
inventories and dynamic reallocation of bike-sharing resources are of
growing interests from both a business and an environmental point of
view. Both of these tasks require accurate predictions of bike rental
behaviors at least on the daily level.

(further motivates \& applications?)

In this project, we strive to answer the following question:

\begin{itemize}
\item
  If we have the past history of bike rental counts as well as records
  of environmental and seasonal conditions, how and how well could we
  predict the bike rental counts in the future?
\item
  In particular, how and how well could we predict for the next whole
  year, and what about for the next few days?
\end{itemize}

Such questions are approached by predictive modeling of daily bike
rental counts from a 2011-2012 Bike Sharing dataset {[}1{]}. The
modeling approach is based on Multiple Linear Regression (MLS), and the
daily bike rental counts are predicted using the environmental variables
(e.g., weather conditions) and seasonal variables (e.g., holiday
schedules) as predictors.

\hypertarget{background}{%
\section{Background}\label{background}}

The aim of this project is to achieve the best model(s) that can be
obtained from past data for the use of predictions for the future,
preferably predictions one year ahead. To validate the prediction power
of models under this setting, the basic goal of this project is to train
all models using only the 2011 data, and then test them on the 2012
data.

The response variable to be predicted is the \textbf{daily} bike rental
count. In the dataset being studied {[}1{]}, the following 3 types of
bike rental counts are recorded:

\begin{enumerate}
\def\labelenumi{\arabic{enumi}.}
\tightlist
\item
  the count of bike rentals by \textbf{casual} users
\item
  the count of bike rentals by \textbf{registered} users
\item
  the \textbf{total} count, which is the sum of casual count and
  registered count.
\end{enumerate}

Two main types of predictors are included in the dataset, the
environmental ones and the seasonal ones:

\begin{enumerate}
\def\labelenumi{\arabic{enumi}.}
\tightlist
\item
  environmental variables
\end{enumerate}

(Table 1: A sample of the data - variable names, meanings, units, sample
values)

\begin{enumerate}
\def\labelenumi{\arabic{enumi}.}
\setcounter{enumi}{1}
\tightlist
\item
  seasonal variables
\end{enumerate}

(Table 2: A sample of the data - variable names, meanings, units, sample
values)

\hypertarget{modeling-analysis}{%
\section{Modeling \& Analysis}\label{modeling-analysis}}

\hypertarget{pre-processing}{%
\subsection{Pre-processing}\label{pre-processing}}

\hypertarget{type-conversion}{%
\subsubsection{Type Conversion}\label{type-conversion}}

To be noticed, the value of categorical variables indicates type labels
and has very limited physical meaning in the magnitude of those values,
which thus cannot be used in the same way as the numeric variables in
MLS models. The categorical variables therefore needs to be recognized
before the actual modeling process and to be carefully handled.

The below variables are interpreted as Boolean variables and are
transformed into \texttt{logical}-type variables in \texttt{R}:

\begin{itemize}
\tightlist
\item
  \texttt{holiday} (holiday or not)
\item
  \texttt{workingday} (working day or not)
\end{itemize}

The below variables are interpreted as categorical variables and are
transformed into \texttt{factor}-type variables in \texttt{R}:

\begin{itemize}
\tightlist
\item
  \texttt{season} (season, from 1 to 4)
\item
  \texttt{yr} (year, from 0 to 1)
\item
  \texttt{mnth} (month, from 1 to 12)
\item
  \texttt{weekday} (weekday, from 0 to 6)
\item
  \texttt{weathersit} (weather type, from 1 to 4)
\end{itemize}

\hypertarget{value-conversion}{%
\subsubsection{Value Conversion}\label{value-conversion}}

The recorded values of \texttt{temp} (measured temperature),
\texttt{atemp} (feeling temperature), \texttt{hum} (measured humidity)
and \texttt{windspeed} (measured wind speed) in the data set being
studied here are the normalized ones; all recorded values are the ones
that have been divided by the maximum of measured values {[}1{]}. For
example, the recorded values of \texttt{temp} (measured temperature) are
obtained by dividing the original measured values by 41 (max) and are
thus all less than or equal to 1.

In this project, these normalized records are scaled back to their
original values for the sake of easier interpretations. For example, the
recorded values of \texttt{temp} (measured temperature) are multiplied
by 41 (max) in the pre-processing process, which recovers the original
scale of temperatures in Celsius.

\hypertarget{variable-selection}{%
\subsection{Variable Selection}\label{variable-selection}}

\hypertarget{response-transformation}{%
\subsubsection{Response Transformation}\label{response-transformation}}

Notably, the behaviors of rental counts from different user types are
considerably different.

\begin{enumerate}
\def\labelenumi{\arabic{enumi}.}
\item
  \textbf{Patterns with weekdays} (see Figure 1,2): Over the time span
  of a week, the casual count usually reaches its minimum in the middle
  of a week (grey dots mostly) and its maximum on weekends (green dots
  mostly), while the registered count does the opposite.
\item
  \textbf{Patterns with temperatures} (see Figure 3): The casual count
  seems more linear in both the feeling and measured temperatures
  (\texttt{atemp} and \texttt{temp}), while the registered count seems
  to be (at least) quadratic.
\end{enumerate}

We therefore expect that the registered counts and casual counts will
follow different distributions and should thus be predicted by separate
models. Furthermore, for the casual count, avoiding unnecessary higher
other terms has the benefit of more stable computations and model
structures. The prediction of total counts will then be obtained by
adding the predicted registered counts and predicted casual counts
together.

\hypertarget{predictor-selection}{%
\subsubsection{Predictor Selection}\label{predictor-selection}}

Given the predictive nature of modeling in the current problem setting,
the predicted response is of greater interests than the actual value of
the parameter estimates, as opposed to that in an inference task. This,
to some degree, relaxes the constraint forbidding colinearity in the
predictors, since colinearity will only lead to instability in the
parameter estimates but not in the predictions; however, we should still
seek to minimize colinearity at least in our beginning model, which
would lead to clearer model structures as well as better
interpretability of model statistics at the early stage of modeling,
which could provide us clearer directions in the improvement process
that follows.

With the above considerations in mind, the predictors in the beginning
model are selected following the 2-step approach below:

\begin{enumerate}
\def\labelenumi{\arabic{enumi}.}
\item
  The scatter plot matrix for the whole set of variables are plotted for
  the 2011 training dataset, and all predictors that seem to be
  significant, i.e., predictors with which the response variable (daily
  rental count, \texttt{cnt}) exhibits a notable visual pattern, are
  selected.
\item
  From the selected predictors above, all the highly correlated
  predictors are removed. Within a group of correlated predictors, only
  the one that has the largest correlation coefficient with the response
  variable as well as having the strongest causal relation with the
  response (in the intuitive sense) will be kept.
\end{enumerate}

It is important that the investigation is done for all predictors for
the sake of minimal loss of information. Note that in practice, the
whole set of predictors is divided into two groups, environmental and
seasonal, and plotted separately, for better readability of the large
scatter plot matrices. The separation is justified by the fact that most
environmental variables, such as weathers, are expected to be
independent of the seasonal variables, such as weekdays and holiday
schedules.

At last, the above process leaves us with a small subset of the very
core predictors for our beginning model: \texttt{weathersit},
\texttt{atemp} and \texttt{weekday}.

\hypertarget{initial-modeling}{%
\subsection{Initial Modeling}\label{initial-modeling}}

\textbf{Beginning Models}

\[\text{cnt \~ workingday + weathersit + atemp + atemp\^{2}}\]

\[ \text{registered ~ workingday + weathersit + atemp + atemp^2}\]

\[ \text{casual ~ workingday + weathersit + atemp}\]

In this process, the model statistics such as p-values are not relied on
as much, because the colinearity issues worsen, which might weaken the
significance of

\textbf{Final Models}

\hypertarget{diagonostic-analysis}{%
\subsection{Diagonostic Analysis}\label{diagonostic-analysis}}

interpretation for the final model as well as residual diagnostics

\hypertarget{validation-and-problemshooting}{%
\subsection{Validation and
Problemshooting}\label{validation-and-problemshooting}}

\hypertarget{refined-model}{%
\subsection{Refined Model}\label{refined-model}}

\hypertarget{prediction-of-the-yearly-growth-ratio}{%
\subsubsection{Prediction of the Yearly Growth
Ratio}\label{prediction-of-the-yearly-growth-ratio}}

The modeling is based on the assumption that the growth trend will
remain the same in the future years as that in the year of 2011. Note
that this is NOT saying that the user base is supposed to remain
unchanged throughout the entire year; the fact that the same scaling
factor works at all points in the entire year is due to the fact that
the MLS model in the later part of the year, e.g., in fall and winter,
are already trained to compensate for rental count growth due to user
growth using the environmental and seasonal variables.

\hypertarget{prediction-without-the-yearly-growth-ratio}{%
\subsubsection{Prediction without the Yearly Growth
Ratio}\label{prediction-without-the-yearly-growth-ratio}}

\hypertarget{prediction}{%
\section{Prediction}\label{prediction}}

\hypertarget{unadjusted-model}{%
\subsection{Unadjusted Model}\label{unadjusted-model}}

\hypertarget{refined-model-1}{%
\subsection{Refined Model}\label{refined-model-1}}

\hypertarget{discussion}{%
\section{Discussion}\label{discussion}}

Models for both long-term and short-term predictions are included.

To be noticed, at least one more year's data is needed for a final
validation of the refined model, which is not available for the moment.
This is to be left for the future work.

Time series

\hypertarget{appendix}{%
\section{Appendix}\label{appendix}}

\hypertarget{preprocessing}{%
\subsection{Preprocessing}\label{preprocessing}}

\hypertarget{type-conversion-1}{%
\subsubsection{Type Conversion}\label{type-conversion-1}}

(codes here)

\hypertarget{value-conversion-1}{%
\subsubsection{Value Conversion}\label{value-conversion-1}}

(codes here)

\hypertarget{variable-selection-1}{%
\subsection{Variable Selection}\label{variable-selection-1}}

\hypertarget{predictors-selection}{%
\subsubsection{Predictors Selection}\label{predictors-selection}}

\hypertarget{predictors-selection-1}{%
\subsubsection{Predictors Selection}\label{predictors-selection-1}}

\hypertarget{response-transformation-1}{%
\subsubsection{Response
Transformation}\label{response-transformation-1}}

\hypertarget{initial-modeling-1}{%
\subsection{Initial Modeling}\label{initial-modeling-1}}

\hypertarget{beginning-model}{%
\subsubsection{Beginning Model}\label{beginning-model}}

\hypertarget{final-model}{%
\subsubsection{Final Model}\label{final-model}}

\hypertarget{diagonostic-analysis-1}{%
\subsection{Diagonostic Analysis}\label{diagonostic-analysis-1}}

\hypertarget{validation-and-problemshooting-1}{%
\subsection{Validation and
Problemshooting}\label{validation-and-problemshooting-1}}

\hypertarget{refined-model-2}{%
\subsection{Refined Model}\label{refined-model-2}}

\hypertarget{prediction-of-the-yearly-growth-ratio-1}{%
\subsubsection{Prediction of the Yearly Growth
Ratio}\label{prediction-of-the-yearly-growth-ratio-1}}

\hypertarget{prediction-without-the-yearly-growth-ratio-1}{%
\subsubsection{Prediction without the Yearly Growth
Ratio}\label{prediction-without-the-yearly-growth-ratio-1}}

\end{document}
